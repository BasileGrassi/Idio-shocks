\documentclass[12pt]{article}
\usepackage[utf8]{inputenc}
%\usepackage[latin1]{inputenc}
\usepackage[T1]{fontenc}
\usepackage[english]{babel}
\usepackage{graphicx}
\usepackage{amsmath}
\usepackage{amsfonts}
\usepackage{amssymb}
\usepackage{fancyhdr}
%\usepackage[nottoc, notlof, notlot]{tocbibind}
\usepackage{geometry}
%\geometry{ hmargin=3.2cm, vmargin=4cm }
%\usepackage{graphicx}
\usepackage{hyperref}

\DeclareMathOperator*{\Max}{Max}
\DeclareMathOperator*{\Argmax}{Argmax}
\newcommand{\E}[1]{\mathbb{E}_{#1}}
\newcommand{\dpp}[2]{\frac{\partial #1}{\partial #2}}
\newcommand{\ddpp}[2]{\frac{d #1}{d #2}}
\newcommand{\ti}[1]{\widetilde{#1}}
\newcommand{\ub}[2]{\underbrace{#1}_{#2}}
\newcommand{\ubt}[2]{\underbrace{#1}_{\text{#2}}}
\newcommand{\oomega}[0]{\overline{\omega}}
\newtheorem{lemma}{Lemma}
\newtheorem{definition}{Definition}
\newtheorem{proposition}{Proposition}
\newtheorem{assumption}{Assumption}

\setlength{\parindent}{0pt}
\setlength{\parskip}{1ex plus 4.ex minus .2ex}
%\linespread{1.2}



\title{Sticky Price, Misallocation and Aggregate Fluctuations}

\author{Basile Grassi\\
\small{\emph{Paris School of Economics and Université Paris 1 Panthéon Sorbonne}}
}

\date{\today}

\begin{document}
\maketitle



\section{Introduction}
Since the work of Restuccia and Rogerson (2008), it is a common view that misallocation generated by distortion in output price is an important factor of TFP differences. Several papers try to understand the underlying factor of distortions both of the output price and the input prices (size dependant taxation, corruption, etc\ldots). In this paper, I (will) study the impact of price stickiness on misallocation and (try) to quantify this effect. 

By imposing a different price that the optimal flexible price, price stickiness generate misallocation. This is the common view of the effect of ``sticky price'' DSGE litterature. However in this literature, the only heterogenity across firms is generate by the last time they have the opportunity to change their price. For a frims with different productivity, size and share in the total output, the distortionnal effect of price stickiness is higher. Moreover  if the price stickiness of firms is different for large and small firms this distortion could have larger effect. 

As show by Buckle and Carlson (2000)\nocite{Buck00}, larger firms update their price more often than small firms. Then it could be that small firms will face more distortions and will be too small compare to their optimal size, whereas large firms will be closer to their optimal size.

Using a close economy version of Melitz (2003)\nocite{Meli03}, I will show that quadratique cost\footnote{Or sticky price \emph{\`a la} Calvo (1983)\nocite{Calv83} or Yun (1996)\nocite{Yun96}} have a non-negligeable effect on aggregate TFP. 

In a dynamic version of this framwork\footnote{Modified to allow firms to face transitory idiosyncratic shocks} the prices stickiness will have an impact on the distribution of firms size and thus in the aggregate fluctuations in the spirit of Gabaix (2011)\nocite{Gaba11} and Carvalho and Gabaix (2010)\nocite{Carv10}.


\bibliographystyle{plain}
\bibliography{bibliotheque.bib}




\end{document}