\documentclass[12pt]{article}
\usepackage[utf8]{inputenc}
%\usepackage[latin1]{inputenc}
\usepackage[T1]{fontenc}
\usepackage[english]{babel}
\usepackage{graphicx}
\usepackage{amsmath}
\usepackage{amsfonts}
\usepackage{amssymb}
\usepackage{fancyhdr}
%\usepackage[nottoc, notlof, notlot]{tocbibind}
\usepackage{geometry}
%\geometry{ hmargin=3.2cm, vmargin=4cm }
%\usepackage{graphicx}
\usepackage{hyperref}

\DeclareMathOperator*{\Max}{Max}
\DeclareMathOperator*{\Argmax}{Argmax}
\newcommand{\E}[1]{\mathbb{E}_{#1}}
\newcommand{\dpp}[2]{\frac{\partial #1}{\partial #2}}
\newcommand{\ddpp}[2]{\frac{d #1}{d #2}}
\newcommand{\ti}[1]{\widetilde{#1}}
\newcommand{\ub}[2]{\underbrace{#1}_{#2}}
\newcommand{\ubt}[2]{\underbrace{#1}_{\text{#2}}}
\newcommand{\oomega}[0]{\overline{\omega}}
\newtheorem{lemma}{Lemma}
\newtheorem{definition}{Definition}
\newtheorem{proposition}{Proposition}
\newtheorem{assumption}{Assumption}

\setlength{\parindent}{0pt}
\setlength{\parskip}{1ex plus 4.ex minus .2ex}
%\linespread{1.2}



\title{Best ideas and inequality}

\author{Basile Grassi\\
\small{\emph{Paris School of Economics and Université Paris 1 Panthéon Sorbonne}}
}

\date{\today}

\begin{document}
\maketitle

\begin{abstract}

\end{abstract}



\section{Introduction}
How people become rich? They build up or inherite a good and profitable firm. Rich people are like Bill Gates or its children. Since, in the data, there are a lot of large firms (strong skewness and fat tailed distribution of firms size), it is possible that the distribution of wealth is also skewed with a fat tailed.

Using a span-of-control model \emph{\`a la} Lucas (1978)\nocite{Luca78} where the underlying heterogeneity is an idea of firms as in Alvarez \emph{et al.} (2008)\nocite{Alva08} and Lucas (2009)\nocite{Luca09}, I (will) show that a fat tailed distribution of firms size and wealth arise by selection of ideas.

Agent with a good (efficient) idea of firms becomes manager and will be able to accumulate assets (because of ``capital spirit'' such as Caroll (2001)\footnote{Cf Kumhof and Rancière (2011)\nocite{Kumh11} for reference.}) whereas agent with a bad idea will only be workers and would not be able to accumulate assets.

This mecanism could generate both a skewed-fat tailed distribution of frims sizes and of wealth inequalities. Introducing redictributive taxes will reduce inequality but generate both misallocation across firms. 

However, with financial constraint in the spirit of Evans and Jovanovic (1989)\nocite{Evan89} and inheritance of ideas in an OLG framework. A redistibutive taxe could have positive effects\footnote{Pas s\^ur de l'intuition mais tout est l\`a.}.


\clearpage

\bibliographystyle{plain}
\bibliography{bibliotheque.bib}




\end{document}